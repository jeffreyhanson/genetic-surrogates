\documentclass[11pt,]{article}
\usepackage{lmodern}
\usepackage{amssymb,amsmath}
\usepackage{ifxetex,ifluatex}
\usepackage{fixltx2e} % provides \textsubscript
\ifnum 0\ifxetex 1\fi\ifluatex 1\fi=0 % if pdftex
  \usepackage[T1]{fontenc}
  \usepackage[utf8]{inputenc}
\else % if luatex or xelatex
  \ifxetex
    \usepackage{mathspec}
    \usepackage{xltxtra,xunicode}
  \else
    \usepackage{fontspec}
  \fi
  \defaultfontfeatures{Mapping=tex-text,Scale=MatchLowercase}
  \newcommand{\euro}{€}
\fi
% use upquote if available, for straight quotes in verbatim environments
\IfFileExists{upquote.sty}{\usepackage{upquote}}{}
% use microtype if available
\IfFileExists{microtype.sty}{%
\usepackage{microtype}
\UseMicrotypeSet[protrusion]{basicmath} % disable protrusion for tt fonts
}{}
\usepackage[margin=1in]{geometry}
\usepackage{graphicx}
\makeatletter
\def\maxwidth{\ifdim\Gin@nat@width>\linewidth\linewidth\else\Gin@nat@width\fi}
\def\maxheight{\ifdim\Gin@nat@height>\textheight\textheight\else\Gin@nat@height\fi}
\makeatother
% Scale images if necessary, so that they will not overflow the page
% margins by default, and it is still possible to overwrite the defaults
% using explicit options in \includegraphics[width, height, ...]{}
\setkeys{Gin}{width=\maxwidth,height=\maxheight,keepaspectratio}
\ifxetex
  \usepackage[setpagesize=false, % page size defined by xetex
              unicode=false, % unicode breaks when used with xetex
              xetex]{hyperref}
\else
  \usepackage[unicode=true]{hyperref}
\fi
\hypersetup{breaklinks=true,
            bookmarks=true,
            pdfauthor={},
            pdftitle={},
            colorlinks=true,
            citecolor=blue,
            urlcolor=blue,
            linkcolor=magenta,
            pdfborder={0 0 0}}
\urlstyle{same}  % don't use monospace font for urls
\setlength{\parindent}{0pt}
\setlength{\parskip}{6pt plus 2pt minus 1pt}
\setlength{\emergencystretch}{3em}  % prevent overfull lines
\setcounter{secnumdepth}{0}

%%% Use protect on footnotes to avoid problems with footnotes in titles
\let\rmarkdownfootnote\footnote%
\def\footnote{\protect\rmarkdownfootnote}

%%% Change title format to be more compact
\usepackage{titling}

% Create subtitle command for use in maketitle
\newcommand{\subtitle}[1]{
  \posttitle{
    \begin{center}\large#1\end{center}
    }
}

\setlength{\droptitle}{-2em}
  \title{}
  \pretitle{\vspace{\droptitle}}
  \posttitle{}
  \author{}
  \preauthor{}\postauthor{}
  \date{}
  \predate{}\postdate{}



\begin{document}

\maketitle


\subsection{Figures}\label{figures}

\begin{figure}[htbp]
\centering
\includegraphics{figures_files/figure-latex/unnamed-chunk-2-1.pdf}
\caption{Summary of single species prioritisations. Single species
prioritisations were generated using amount-based targets, amount-based
and surrogate-based targets, and amount-based and genetic-based targets
for each species. Data shows the performance of prioritisations
generated using these three sets of targets. Bars denote means and
standard errors.}
\end{figure}

\begin{figure}[htbp]
\centering
\includegraphics{figures_files/figure-latex/unnamed-chunk-3-1.pdf}
\caption{Multi-species prioritisations. Panel (a) shows the
prioritisation generated for using just amount-based targets. Panel (b)
shows the prioritisation generated using amount-based and surrogate
based targets. Panel (c) shows the prioritisation generated using
amount-based and genetic-based targets}
\end{figure}

\begin{figure}[htbp]
\centering
\includegraphics{figures_files/figure-latex/unnamed-chunk-4-1.pdf}
\caption{Summary of multi-species prioritisations. Three prioritisations
were generated using amount-based targets, amount-based and
surrogate-based targets, and amount-based and genetic-based targets for
each species. Data shows the performance of these prioritisations based
on how much genetic variation they explain. Bars denote means and
standard errors.}
\end{figure}

\begin{figure}[htbp]
\centering
\includegraphics{figures_files/figure-latex/unnamed-chunk-5-1.pdf}
\caption{The relationship between surrogates and genetic variation
secured in prioritizations.}
\end{figure}

\end{document}
